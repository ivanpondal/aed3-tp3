\section{Aplicaciones del problema en la ciencia molecular}

% El Doc ha estudiado este problema durante mucho tiempo, y cree entender que
% se puede resolver gracias a las aplicaciones prácticas de la teoría de
% grafos que describen en su paper Ehrlich y Rarey [1]. Está convencido de que
% con resolver MCS le alcanza para rediseñar correctamente la nueva componente
% en cuestión, pero Marty desconfía.
% a) Su tarea consiste en leer este paper y explicarle a Marty qué problema de
%    grafos se buscar resolver en las aplicaciones del paper, encontrar las
%    diferencias entre dicho problema y MCS, y explicar qué aplicaciones tiene
%    este problema en la química. No es parte de este TP resolver el problema
%    del paper, en todo caso es problema del Doc.

Entre las diversas aplicaciones que tiene el problema de \acr{MCS}, cabe
destacar las relacionadas con el campo de la química y la biología molecular,
en las que pone el foco el trabajo de Ehrlich y Rarey\cite{ehrlich}. Los
grafos constituyen una forma muy conveniente e intuitiva de representar la
estructura de una molécula, con la ventaja agregada de poder aprovechar el
gran conocimiento computacional existente sobre el tema. En este contexto,
resolver el problema de \acr{MCS} resulta de gran utilidad para buscar
subestructuras comunes entre dos moléculas dadas, o cuantificar la similitud
entre las mismas.

El artículo engloba, bajo el nombre general de \acr{MCS}, a dos problemas
similares, denominados \acr{MCES} (\emph{maximum common edge subgraph}) y
\acr{MCIS} (\emph{maximum common induced subgraph}). El primero de ellos
coincide con el problema que se estudiará en este trabajo: dados dos grafos,
encontrar el subgrafo común a ambos con máxima cantidad de aristas. El
segundo, en cambio, consiste en encontrar un grafo con la mayor cantidad
posible de nodos que sea isomorfo, simultáneamente, a un subgrafo inducido de
otros dos grafos.

Dentro del contexto de aplicación estudiado por el paper, los grafos
utilizados presentan ciertas particularidades. Con frecuencia, al modelar
estructuras moleculares, se utilizan etiquetas para indicar propiedades de los
átomos o enlaces representados por cada uno de los nodos o aristas,
respectivamente. Luego, cuando se desea resolver \acr{MCS} para encontrar
subestructuras comunes a dos moléculas, solo pueden mapearse átomos o enlaces
cuyas etiquetas coincidan, agregando una nueva restricción al problema que no
está presente en la versión que será abordada en este trabajo.

En el trabajo citado, se hace un recorrido por diversos escenarios en los que
resulta aplicable \acr{MCS}, la mayoría de ellos en el área de la biología
molecular. Se mencionan dos grandes categorías de problemas: los relacionados
con pequeñas moléculas orgánicas, como las drogas, y los concernientes a
moléculas grandes, por ejemplo, las proteínas. Algunos de los incluidos en el
primer grupo son la obtención de métricas de similitud estructural que
permiten la clasificación de compuestos en familias o la predición de sus
efectos activos, el mapeo de átomos y enlaces entre reactivos y productos en
reacciones, y la elaboración de modelos matemáticos para cuantificar
la relación entre estructura y la actividad de compuestos químicos. En el
segundo grupo, se puede mencionar tanto la búsqueda de semejanzas
estructurales entre moléculas a escala global como la identificación de
similitudes de carácter local; estas útimas resultan de utilidad para procesar
grandes bases de datos y elaborar hipótesis acerca del origen evolutivo y las
funciones desempeñadas por las proteínas.
