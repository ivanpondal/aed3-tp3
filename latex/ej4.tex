\section{Heurística golosa}

% Desgraciadamente, el problema es tan difícil que ni Marty ni el Doc conocen
% una manera de resolverlo en tiempo polinomial para el caso general. Para no
% quedarse con las manos vacías, Marty les pide diseñar e implementar una
% heurística constructiva golosa para MCS y desarrollar los siguientes puntos
% para conocer la calidad de las soluciones que le proveerán:
% a) Explicar detalladamente el algoritmo implementado.
% b) Calcular el orden de complejidad temporal de peor caso del algoritmo.
% c) Describir instancias de MCS para las cuales la heurística no proporciona
%    una solución óptima. Indicar qué tan mala puede ser la solución obtenida
%    respecto de la solución óptima.
% d) Realizar una experimentación que permita observar la performance del
%    algoritmo en términos de tiempo de ejecución en función del tamaño de
%    entrada.

\subsection{Experimentación}

\pgfplotstableread[header=false]{../exp/ej4/known_sol_greedy_exp}\knowngreedy
\pgfplotstableread[header=false]{../exp/optimal_solutions}\optimalsolutions

\pgfplotstableset{
	create on use/sol/.style={copy column from table={\optimalsolutions}{[index] 1}}
}

\pgfplotstabletypeset[
	every head row/.style={
		after row=\hline
	},
	columns={0, sol, 1, 3, 4},
	columns/0/.style={
		column name=\textsc{Instancia},
		column type={l},
		string replace*={_}{\_},
		string type,
		assign cell content/.code={
			\pgfkeyssetvalue{/pgfplots/table/@cell content}{\texttt{##1}}
		}
	},
	columns/sol/.style={
		column name=$\#E(MCS)$,
		int detect
	},
	columns/1/.style={
		column name=$\#E(H)$,
		int detect
	},
	columns/3/.style={
		column name=$T_{\mu}$,
	},
	columns/4/.style={
		column name=$T_{\sigma}$,
	}
]\knowngreedy
