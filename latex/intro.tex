\section*{Introducción}
\addcontentsline{toc}{section}{Introducción}

En el presente trabajo, se explorarán distintos alogoritmos para la resolución
del problema de grafos conocido como \emph{máximo subgrafo común}, o
\acr{MCS} (\emph{maximum commom subgraph}). Este problema consiste en
encontrar, dados dos grafos cualesquiera, un tercer grafo que sea,
simultáneamente, isomorfo a algún subgrafo de los dos primeros, y con la
máxima cantidad posible de aristas.

Más formalmente, si $G_1 = (V_1, E_1)$ y $G_2 = (V_2, E_2)$ son dos grafos
simples, el problema de máximo subgrafo común consiste en encontrar un grafo
$H = (V_H, E_H)$, isomorfo tanto a un subgrafo de $G_1$ como a un subgrafo
de $G_2$ que maximice $\#E(H)$.

Encontrar el \acr{MCS} entre dos grafos es un problema NP-
completo\cite{garey}, por lo que, a menos que $\text{P} = \text{NP}$, no
existen algoritmos que permitan resolver el caso general en tiempo polinomial.
Por este motivo, el problema es abordado en este trabajo desde dos enfoques
distintos, según se favorezca la exactitud o el rendimiento temporal de las
soluciones.

En primer lugar, se presentan dos algoritmos exactos: uno de complejidad
temporal exponencial que resuelve el caso general aplicando la técnica de
\emph{backtracking}, y otro restringido a un caso particular del problema
(cuando se tienen como entrada un grafo completo y un \emph {cografo}), cuya
complejidad temporal es polinomial. A continuación, se exploran tres
soluciones heurísticas (heurística golosa, heurística de búsqueda local, y
metaheurística \emph{tabu search}), que renuncian a la exactitud del enfoque
previo en pos de lograr una complejidad polinomial para el caso general. Por
último, se realiza un comparativo entre los resultados obtenidos para
distintos tipos de grafos utilizando cada una de las técnicas, extrayendo
conclusiones acerca de la calidad y la eficiencia de las mismas.

\subsection*{Consideraciones generales}
\addcontentsline{toc}{subsection}{Consideraciones generales}

\begin{enumerate}[label=(\roman*)]
\item Es importante notar que el problema de \acr{MCS} admite una formulación
equivalente, donde el subgrafo común buscado, $H$, es visto como un mapeo
biyectivo entre un subconjunto de los nodos de $G_1$ y un subconjunto de los
nodos de $G_2$. Más aún, suponiendo sin pérdida de generalidad que $\#V(G_1)
\leq \#V(G_2)$, esto es, que $G_1$ tiene una cantidad de nodos menor o igual
que la de $G_2$, el mapeo puede tomarse considerando la totalidad de los nodos
de $G_1$. Esto se debe a que, si ya se cuenta con una solución cuya cantidad
de aristas es máxima y que no utiliza todos los nodos de $G_1$, esta puede
extenderse para incluir los nodos faltantes sin perjudicar su optimalidad,
siendo los mismos mapeados a nodos arbitarios de $G_2$.

Para formalizar un poco más esta perspectiva, sean $G_1$ y $G_2$ tales que
$\#V(G_1) \leq \#V(G_2)$, y $H$ una solución de \acr{MCS} con $\#V(H) =
\#V(G_1)$. Entonces, existen un mapeo biyectivo entre los nodos de $H$ y los
de $G_1$, y un mapeo inyectivo entre los nodos de $H$ y los de $G_2$. Esto nos
permite definir $s : V(G_1) \to V (G_2)$, de forma tal que a todo $v \in
V(G_1)$ le corresponda el mismo nodo de $H$ que a $s(v) \in V(G_2)$; visto de
otro modo, cada nodo de $H$ se corresponde con un par de la forma $(v, s(v))$.

Más aún, conociendo $s$, las aristas de $H$ quedan unívocamente determinadas:
si dos nodos de $H$ correspondientes a los pares $(v, s(v))$ y $(w, s(w))$,
respectivamente, están unidos por una arista, entonces para que $H$ sea una
solución válida de \acr{MCS}, $(v, w)$ deberá ser una arista de $G_1$ y
$(s(v), s(w))$ deberá ser una arista de $G_2$. Recíprocamente, si tanto $v$ y
$w$ como $s(v)$ y $s(w)$ son adyacentes en sus respectivos grafos, los nodos
correspondientes en $H$ a cada par deberán ser adyacentes, o el número de
aristas de $H$ no sería máximo. De todo lo anterior, se sigue que para dar una
solución a \acr{MCS} basta con exhibir un mapeo $s : V(G_1) \to V (G_2)$
apropiado.

Dado que esta interpretación del problema simplifica la formulación
e implementación de muchos de los procedimientos expuestos en este trabajo,
será utilizada asiduamente a lo largo del mismo.

\item Todos los algoritmos presentados en el trabajo fueron implementados en
el lenguaje C++, utilizando en varios casos funciones y estructuras de datos
provistas por la biblioteca estándar del lenguaje. Estas implementaciones
fueron sometidas a diversas pruebas experimentales, durante las cuales se
tuvieron en cuenta las siguientes consideraciones generales:

\begin{itemize}
\item Solo se midió el costo temporal de generar las soluciones,
omitiendo el demandado para la lectura y escritura de los datos.
\item Para la medición de los tiempos de ejecución, se utilizaron los
métodos proporcionados por el \emph{header} \texttt{chrono} de la
biblioteca estándar de C++. Los tiempos fueron medidos en nanosegundos.
% \item Todas las pruebas se corrieron en las computadoras del Laboratorio 4
% del Departamento de Computación (\acr{FCEN} - \acr{UBA}).
% \item Todos los experimentos aleatorios se realizaron con semilla de valor 7.
\end{itemize}

\end{enumerate}

\subsection*{Instancias de prueba}
\addcontentsline{toc}{subsection}{Instancias de prueba}

Con el fin de tener un método de comparación entre los diversos algoritmos
implementados, se generon una serie de instancias con características distintas
para el problema a resolver. Estas constan de pares de grafos $(G_1,
G_2)$ y están divididas en dos grupos que son: instancias con solución óptima
conocida e instancias con solución desconocida. La idea fue tener un conjunto
que cubriera un número razonable de escenarios con los cuales poder
experimentar y tener un aproximación de cómo se comporta cada heurísitca en los
mismos.

\subsubsection*{Instancias con solución conocida}

Dada la naturaleza del problema de buscar el máximo subgrafo común, a la hora de
crear instancias con las cuales poder medir rendimiento y calidad de soluciones
resulta de mucho interés poder conocer cuál es el resultado de la solución
óptima. Es así como para este conjunto de elementos de prueba se pensaron las
siguientes familias que por el modo en el cual están creadas permiten conocer
cuál sería el número de aristas que aporta el máximo subgrafo común.

\begin{itemize}
	\item Familias para algoritmos exactos

	En este trabajo se desarrollaron dos algoritmos exactos: uno resuelve
	mediante \emph{backtracking} mientras que el otro resuelve en
	tiempo polinomial el problema entre un cografo y un grafo completo. Teniendo
	estas implementaciones se definió y resolvió el siguiente conjunto de
	instancias.

	Solución obtenida mediante \emph{backtracking}:
	\begin{itemize}
		\item \texttt{aleatorio\_n5\_c025}:

			$N_1 = N_2 = 5$ con $25\%$ de probabilidad de que exista una arista
			entre dos nodos.
		\item \texttt{aleatorio\_n5\_c050}:

			$N_1 = N_2 = 5$ con $50\%$ de probabilidad de que exista una arista
			entre dos nodos. La solución es obtenida mediante \emph{backtracking}.
		\item \texttt{aleatorio\_n5\_c075}:

			$N_1 = N_2 = 5$ con $75\%$ de probabilidad de que exista una arista
			entre dos nodos. La solución es obtenida mediante \emph{backtracking}.
	\end{itemize}

	Solución obtenida mediante implementación para cografos contra completos:
	\begin{itemize}
		\item \texttt{cografo\_n100\_k50}:

			$G_1$ un cografo de 100 nodos y $G_2$ un completo $K_{50}$.
		\item \texttt{cografo\_n50\_k100}:

			$G_1$ un cografo de 50 nodos y $G_2$ un completo $K_{100}$.
	\end{itemize}

	\item Familias donde $G_1$ es subgrafo de $G_2$

	Uno de los métodos más sencillos de conocer el máximo subgrafo común es
	construyendo $G_1$ y tomando $G_2 = G_1$ para luego agregarle vértices y
	aristas de forma tal que $G_1 \neq G_2$ pero $G_1$ subgrafo de $G_2$.

	\begin{itemize}
		\item \texttt{aleatorio\_subgrafo\_n500\_c025}

			$N_1 = 500$ y $N_2 = 1000$ con $25\%$ de probabilidad de que exista
			una arista entre dos nodos.
		\item \texttt{aleatorio\_subgrafo\_n500\_c050}

			$N_1 = 500$ y $N_2 = 1000$ con $50\%$ de probabilidad de que exista
			una arista entre dos nodos.
		\item \texttt{aleatorio\_subgrafo\_n500\_c075}

			$N_1 = 500$ y $N_2 = 1000$ con $75\%$ de probabilidad de que exista
			una arista entre dos nodos.
		\item \texttt{arbol\_subgrafo\_n500}

			$N_1 = 500$ y $N_2 = 1000$ con $G_1$ árbol generados de forma
			aleatoria y $G_2$ árbol incremental sobre el anterior.
		\item \texttt{completo\_subgrafo\_n500}

			Completo $K_{500}$ contra $K_{1000}$.
		\item \texttt{ciclo\_subgrafo\_n500}

			Ciclo $C_{500}$ contra $C_{1000}$.
		\item \texttt{aleatorio\_bipartito\_vs\_completo\_n500\_k500\_c050}

			Bipartito con dos particiones de tamaño 500 teniendo $50\%$ de probabilidad de que exista una
			arista entre ellas, contra bipartito completo $K_{500,500}$.
		\item \texttt{aleatorio\_bipartito\_vs\_completo\_n250\_k500\_c050}

			Bipartito con una partición de 250 y otra de 500 teniendo $50\%$ de probabilidad de que exista una
			arista entre ellas, contra bipartito completo $K_{250,500}$.
		\item \texttt{bosque\_vs\_completo\_n1000\_d10}

			Bosque con $N_1 = 1000$ y 10 componentes conexas contra $G_2$ con 10
			$K_{100}$.
		\item \texttt{aleatorio\_vs\_completo\_n1000\_d10\_c050}

			$N_1 = 1000$ donde ademas de asegurar la existencia de 10
			componentes conexas, dos nodos de una misma componente tienen una
			arista con $50\%$ de probabilidad, contra $G_2$ con 10 $K_{100}$.
	\end{itemize}
\end{itemize}

\subsubsection*{Instancias con solución desconocida}

A pesar del valor que suma conocer la solución al problema para los casos
generados, uno podría estar sesgando mucho la experimentación al probar
únicamente los casos construidos a medida. Es por esto que se decidió también
poseer un conjunto de pruebas donde no se conoce la solución óptima, pero que
son útiles para comparar los algoritmos desarrollados entre sí.

\begin{itemize}
	\item \texttt{aleatorio\_n1000\_c025}

		$N_1 = N_2 = 1000$ con $25\%$ de probabilidad de que exista una arista
		entre dos nodos.
	\item \texttt{aleatorio\_n1000\_c050}

		$N_1 = N_2 = 1000$ con $50\%$ de probabilidad de que exista una arista
		entre dos nodos.
	\item \texttt{aleatorio\_n1000\_c075}

		$N_1 = N_2 = 1000$ con $75\%$ de probabilidad de que exista una arista
		entre dos nodos.
	\item \texttt{arbol\_n1000}

		$N_1 = N_2 = 1000$ dos árboles generados de forma aleatoria.
	\item \texttt{aleatorio\_bipartito\_n500\_k500\_c050}

		Dos grafos bipartitos con particiones de tamaño 500 teniendo $50\%$ de
		probabilidad de que exista una arista entre sus particiones.
	\item \texttt{aleatorio\_bipartito\_n250\_k500\_c050}

		Dos grafos bipartitos con particiones de tamaño 250 y 500 teniendo $50\%$ de
		probabilidad de que exista una arista entre sus particiones.
	\item \texttt{bosque\_n1000\_d10\_v025}

		Dos bosques con $N_1 = N_2 = 1000$ y 10 componentes conexas.
	\item \texttt{aleatorio\_n1000\_d10\_c050\_v025}

		$N_1 = N_2 = 1000$ asegurando 10 componentes conexas en cada grafo con
		una probabilidad de $50\%$ de aristas entre dos nodos de una misma
		componente conexa.
\end{itemize}
