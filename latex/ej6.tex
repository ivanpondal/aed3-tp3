\section{Metaheurística \emph{tabu search}}

% A esta altura ya se cansaron de escuchar las peleas entre Marty y el Doc,
% con lo cual les dejaron muy en claro que lo último que les ofrecen es
% implementar una metaheurística para resolver el problema. Por suerte, fueron
% tomados en serio, así que evaluaron todos juntos la situación y decidieron
% que lo mejor es utilizar la metaheurística Taboo Search. Para ponerle punto
% final a esta historia, deberán diseñar e implementar un algoritmo para MCS
% que incluya la susodicha metaheurística, y desarrollar los siguientes puntos
% que validen la solución encontrada, para que finalmente reine el amor y la
% igualdad entre Marty y el Doc:
% a) Explicar detalladamente el algoritmo implementado. Plantear distintos
%    criterios de parada y de tamaño de la lista taboo de la heurística.
% b) Realizar una experimentación que permita observar los tiempos de
%    ejecución y la calidad de las soluciones obtenidas. Se debe experimentar
%    variando los valores de los parámetros de la metaheurística (tamaño de la
%    lista taboo, criterios de parada, etc.) y las vecindades utilizadas en la
%    búsqueda local. Elegir, si es posible, la configuración que mejores
%    resultados provea para el grupo de instancias utilizado.

\subsection{Introducción}
Para mejorar los resultados obtenidos mediante la heurística de búsqueda
local, se implementó una metaheurística \emph{tabu search}. Esta
metaheurística tiene como objetivo realizar una ejecución guiada de la
heurística de búsqueda local, evitando el estancamiento en óptimos locales;
para esto, se continúan explorando soluciones incluso aunque resulten menos
óptimas, con la esperanza de encontrar mejores resultados a largo plazo. Para
evitar la generación de ciclos por un retorno constante a los óptimos locales
ya encontrados, se utiliza una \emph{lista tabú}, donde se almacenan
soluciones o características de soluciones que no deberían volver a visitarse.
También se puede almacenar información adicional, por ejemplo, una valoración
de las soluciones consideradas \emph{tabú} que permita decidir entre ellas en
caso de no existir alternativa.

\subsection{Características}
Dado que la metaheurística no es más que un esquema algorítmico general, para
definir la implementación de la metaheurística, fue necesario tomar decisiones
sobre varios puntos, que serán expuestos a continuación.

\subsubsection{Estructura del algoritmo}

\subsubsection{Lista tabú}

\subsubsection{Criterio de eliminación}

\subsubsection{Criterio de parada}

\subsubsection{Función de aspiración}

\subsection{Experimentación}
