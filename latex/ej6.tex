\section{Metaheurística \emph{tabu search}}

% A esta altura ya se cansaron de escuchar las peleas entre Marty y el Doc,
% con lo cual les dejaron muy en claro que lo último que les ofrecen es
% implementar una metaheurística para resolver el problema. Por suerte, fueron
% tomados en serio, así que evaluaron todos juntos la situación y decidieron
% que lo mejor es utilizar la metaheurística Taboo Search. Para ponerle punto
% final a esta historia, deberán diseñar e implementar un algoritmo para MCS
% que incluya la susodicha metaheurística, y desarrollar los siguientes puntos
% que validen la solución encontrada, para que finalmente reine el amor y la
% igualdad entre Marty y el Doc:
% a) Explicar detalladamente el algoritmo implementado. Plantear distintos
%    criterios de parada y de tamaño de la lista taboo de la heurística.
% b) Realizar una experimentación que permita observar los tiempos de
%    ejecución y la calidad de las soluciones obtenidas. Se debe experimentar
%    variando los valores de los parámetros de la metaheurística (tamaño de la
%    lista taboo, criterios de parada, etc.) y las vecindades utilizadas en la
%    búsqueda local. Elegir, si es posible, la configuración que mejores
%    resultados provea para el grupo de instancias utilizado.

\subsection{Introducción}
Para mejorar los resultados obtenidos mediante la heurística de búsqueda
local, se implementó una metaheurística \emph{tabu search}. Esta
metaheurística tiene como objetivo realizar una ejecución guiada de la
heurística de búsqueda local, evitando el estancamiento en óptimos locales;
para esto, se continúan explorando soluciones incluso aunque resulten menos
óptimas, con la esperanza de encontrar mejores resultados a largo plazo. Para
evitar la generación de ciclos por un retorno constante a los óptimos locales
ya encontrados, se utiliza una \emph{lista tabú}, donde se almacenan
soluciones o características de soluciones que no deberían volver a visitarse.
También se puede almacenar información adicional, por ejemplo, una valoración
de las soluciones consideradas \emph{tabú} que permita decidir entre ellas en
caso de no existir alternativa.

\subsection{Características}
Dado que la metaheurística no es más que un esquema algorítmico general, para
definir la implementación de la metaheurística, fue necesario tomar decisiones
sobre varios puntos, que serán expuestos a continuación.

\subsubsection{Estructura del algoritmo}
La metaheurística se implementó utilizando como base la heurística de búsqueda
local presentada en la sección anterior, manteniendo la posibilidad de
elegir cuál de las dos vecindades considerar. El esquema de ejecución es muy
sencillo:
\begin{enumerate}
    \item Se ejecuta la heurística golosa, para utilizar la solución obtenida
    como punto de partida.
    \item Se ejecuta una iteración de la heurística de búsqueda local, con la
    vecindad elegida. De esta forma se determina un \emph{movimiento} hacia
    una solución vecina, que para ambas vecindades puede representarse
    mediante un par de vértices de $G_2$ involucrados en la transformación: en
    el caso de la vecindad I o de la operación \emph{exchange} de la vecindad
    II, uno de los vértices pertenece al mapeo de la solución actual y el otro
    no, mientras que en la operación \emph{swap} de la vecindad II ambos nodos
    pertenecen al mapeo.
    \item Se agrega a la lista tabú el movimiento inverso al realizado, para
    indicar que no se desea revertirlo en el futuro.
    \item Si no se cumple el criterio de parada elegido, se repite el paso 2,
    pero prohibiendo realizar movimientos que se encuentren en la lista tabú.
    Si se cumple el criterio de parada, se devuelve la mejor solución
    encontrada hasta el momento.
\end{enumerate}

La diferencia entre este esquema metaheurístico y los algoritmos de búsqueda
local es que, si al explorar la vecindad, todas las soluciones vecinas
evaluadas resultan ser peores que la actual, se permite realizar un movimiento
hacia una de ellas, que es elegida de forma aleatoria.

\subsubsection{Criterio de eliminación}
Dado que la capacidad de la lista tabú es limitada, se hace necesario contar
con un criterio para decidir qué elementos eliminar de la misma cuando no hay
espacio para almacenar movimientos nuevos. La opción más sencilla es,
posiblemente, adoptar un enfoque \acr{FIFO}: cuando hay que eliminar un
elemento, se selecciona el que más tiempo lleva en la lista. Sin embargo, esto
puede ocasionar que movimientos que dieron muy buenos resultados (y que por
lo tanto, no se desea revertir) sean rápidamente desplazados de la lista por
movimientos menos significativos, lo cual no es deseable.

Para solucionar este inconveniente, se decidió adoptar un criterio de
eliminación híbrido. A cada elemento de la lista tabú se le asocia un peso, y
cada vez que debe descartarse uno de ellos, se selecciona aquel cuyo peso es
menor. El peso de un movimiento dado es directamente proporcional a la
cantidad de aristas aportadas, e inversamente proporcional al tiempo que lleva
en la lista, pudiendo regularse la influencia de cada uno de estos factores en
el cálculo mediante un parámetro $c \in [0,1]$. La fórmula utilizada para
calcular el peso de un movimiento, en función de la cantidad $a$ de aristas
aportadas y de la cantidad $t$ de iteraciones que lleva en la lista tabú, es
\[ P(t,a,c) = c \left( \frac{a + m} {2 m} \right) + (1-c) \left(\frac{1}{1+0.1t} \right) \]
donde:
\begin{itemize}
    \item $m = \max_{v \in V(G_1)} \deg(v)$. Este valor fue seleccionado
    porque, en cualquiera de las dos vecindades planteadas para la búsqueda
    local, un movimiento cualquiera puede agregar o restar, a lo sumo, $m$
    aristas de la solución. De esta forma $-m < a < m$, por lo que el valor
    del primer término de la fórmula anterior resulta ser un número entre $0$
    y $1$.
    \item El segundo término de la suma es también un valor entre $0$ y $1$, y
    además el valor del mismo disminuye cada vez más lentamente a medida que
    pasa el tiempo. Esto favorece que los movimientos que aportan poca
    cantidad de aristas sean rápidamente descartados de la lista, mientras que
    los que resultan más significativos desarrollan cierta ``resistencia'' a
    ser eliminados.
\end{itemize}

Notar que, con esta definición, el valor del peso para un movimiento
determinado es siempre un valor entre $0$ y $1$. Cada vez que se agrega un
nuevo elemento a la lista tabú, se calcula su peso con $t = 0$, y los pesos de
todos los demás elementos son actualizados sumando $1$ al valor anterior de
$t$. La lista, por su parte, es una cola de prioridad donde la prioridad de
cada elemento es su peso; esto permite seleccionar en tiempo constante el
próximo elemento a ser eliminado, y actualizar el peso de todos los demás en
un tiempo lineal en el tamaño de la lista.

\subsubsection{Criterio de parada}
Otro de los elementos a definir fue qué criterio utilizar para decidir cuándo
dejar de iterar. Se plantearon dos alternativas posibles, pudiendo utilizarse
solo una de ellas o ambas en simultáneo, en cuyo caso el algoritmo se detendrá
cuando cualquiera de los dos criterios se cumpla.

\begin{enumerate}
    \item \textbf{Cantidad de iteraciones}: el algoritmo se detiene luego de
    ejecutar una cantidad determinada de iteraciones de la búsqueda local.
    \item \textbf{Cantidad de iteraciones sin mejora}: el algoritmo se detiene
    tras ejecutar una determinada cantidad de iteraciones de la búsqueda local
    sin encontrar una solución mejor que la mejor solución encontrada hasta el
    momento.
\end{enumerate}

En ambos casos, se entiende por iteración a explorar la vecindad de una
solución; en otras palabras, cada movimiento hacia una solución vecina
corresponde a una iteración.

\subsubsection{Función de aspiración}
En ocasiones, el resultado de realizar un determinado movimiento puede ser tan
bueno que resulte deseable realizarlo incluso aunque aparezca en la lista
tabú. A esto se lo conoce como \emph{función de aspiración}, y tiene como
parámetro un umbral configurable $\alpha \in [0, 1]$. De esta forma, si la
cantidad de aristas $a$ que tendría la solución resultante de realizar un
determinado movimiento es mayor que el producto entre el umbral y la cantidad
$a^*$ de aristas de la mejor solución hallada hasta el momento, es decir, si
$a > \alpha a^*$, entonces el movimiento se realiza independientemente de que
se encuentre o no en la lista tabú.

\subsection{Experimentación}

Para la experimentación sobre la implementación de esta metaheurística se
llevó a cabo una comparación entre los resultados obtenidos de variar los
distintos parámetros de la misma (tamaño de la lista taboo, criterios de
parada, funcion de aspiración). La comparación consta de contrastar la
cantidad de aristas encontrada y los tiempos de ejecución del algoritmo.
Además se exhibe como referencia la cantidad maxima de aristas.

En cuanto a los parametros intrínsecos a la búsqueda local, se los
fijó para diferentes valores, y sobre cada uno de ellos se varió los de
la metaheurística.

El grupo de instancias utilizado corresponde a un conjunto de grafos de varias
familias distintas.

\begin{itemize}
\item Cografo vs. cografo
\item Grafo aleatorio vs. subgrafo del mismo (variando las densidades de
aristas)
\item Árbol vs. subgrafo
\item Completo vs. subgrafo
\item Ciclo vs. subgrafo
\item Aleatorio bipartito vs. completo (variando cantidad de nodos)
\item Aleatorio vs. completo (variando cantidad de nodos)
\end{itemize}

Se consideró que este conjunto de instancias resultaba representativo en
cuanto al espectro a de grafos a explorar.

Con el fin de lograr una mayor precisión en los datos, el algoritmo fue
ejecutado 10 veces para cada instancia y combinación de parámetros, y de esos
datos se calcula el promedio. Los datos que se exhiben corresponden a los
promedios de las aristas encontradas por el algoritmo para cada combinación de
parámetros.

Se hará uso de la siguiente notación:

\begin{itemize}
\item $MCS$: Cantidad de aristas de la solucíon óptima.
\item $LS$: Cantidad de aristas encontradas con el algoritmo de búsqueda
local.
\item $i$\textbf{num}: Cantidad de aristas encontradas por la metaheurística
con parámetro de cantidad total de iteraciones = \textbf{num}. En el caso de
$INF$, se considera que no hay límite de iteraciones.
\item $a$\textbf{umbral}: Cantidad de aristas encontradas por la
metaheurística con parámetro de función de aspiración = \textbf{umbral}.
\item $ng$\textbf{límite}: Cantidad de aristas encontradas por la
metaheurística con parámetro de límite de iteraciones sin registrar ganancia =
\textbf{límite}
\item $T(*)$: Tiempo de ejecución de $*$ medido en nanosegundos.
\item $\Delta \%$: Diferencia porcentual entre un caso y otro.
\end{itemize}

En la siguiente experimentación se varió el umbral de la función de
aspiración. Los valores que se mantuvieron fijos fueron:

\begin{itemize}
\item Tipo de vecindad = \textit{exchange}
\item Proporción de vecinos considerada = 1\%
\item Tamaño de lista tabú = 600
\item Cantidad de iteraciones = 1000
\item Cantidad de iteraciones sin mejora = 1000 (equivalente a no tener en
cuenta este parámetro ya que tiene el mismo valor que la cantidad de
iteraciones total)
\item Relacion entre valor del peso de aristas vs. tiempo = 0.5 (el tiempo que
lleva un movimiento en la lista tabú y la cantidad de aristas que proporciona
tienen igual importancia)
\end{itemize}

A continuación se muestra una tabla exponiendo los datos obtenidos.

\pgfplotstableread[header=false]{../exp/small_optimal_solutions}{\optimalsolutions}
\pgfplotstableread[header=false]{../exp/ej6/known_sol_greedy_exp}{\knowngreedy}
\pgfplotstableread[header=false]{../exp/ej6/known_sol_local_search_exp}{\knownlocalsearch}

% \pgfplotstableread[header=false]{../exp/ej6/1-0001-1000-0.01-0100-0.5-0.1-1.0}{\lowtaboolist}
% \pgfplotstableread[header=false]{../exp/ej6/1-0060-1000-0.01-0100-0.5-0.1-1.0}{\hightaboolist}
% \pgfplotstableread[header=false]{../exp/ej6/1-0600-1000-0.01-0100-0.5-0.1-1.0}{\highertaboolist}

\pgfplotstableread[header=false]{../exp/ej6/1-0600-1000-0.01-1000-0.5-0.1-0.3}{\lowiterationlowaspiration}
\pgfplotstableread[header=false]{../exp/ej6/1-0600-1000-0.01-1000-0.5-0.1-0.8}{\lowiterationhighaspiration}
\pgfplotstableread[header=false]{../exp/ej6//1-0600-1000-0.01-1000-0.5-0.1-0.8}{\lowiterationfullaspiration}

% \pgfplotstableread[header=false]{../exp/ej6/1-0600-4000-0.01-0100-0.5-0.1-0.8}{\mediumiteration}
\pgfplotstableread[header=false]{../exp/ej6/1-0600-2000-0.01-2000-0.5-0.1-0.3}{\highiterationlowaspiration}
\pgfplotstableread[header=false]{../exp/ej6/1-0600-2000-0.01-2000-0.5-0.1-0.8}{\highiterationhighaspiration}
\pgfplotstableread[header=false]{../exp/ej6/1-0600-2000-0.01-2000-0.5-0.1-1.0}{\highiterationfullaspiration}

\pgfplotstableread[header=false]{../exp/ej6/1-0600-0000-0.01-2000-0.5-0.1-0.3}{\nolimititerationlowaspiration}

\pgfplotstableread[header=false]{../exp/ej6/1-0600-4000-0.01-0100-0.5-0.1-0.8}{\higheriterationlowernogainhighaspiration}
% \pgfplotstableread[header=false]{../exp/ej6/1-0060-2000-0.01-2000-0.5-0.1-0.3}{\lowtaboohighiteration}

\pgfplotstableread[header=false]{../exp/ej6/1-0060-2000-0.01-100-0.5-0.1-0.3}{\highiterationlowernogainlowaspiration}
\pgfplotstableread[header=false]{../exp/ej6/1-0600-2000-0.01-2000-0.5-0.1-0.3}{\highiterationhighnogainlowaspiration}
% \pgfplotstableread[header=false]{../exp/ej6/1-0600-2000-0.01-800-0.5-0.1-0.3}{\highiterationlownogainlowaspiration}

\pgfplotstablecreatecol[copy column from table={\optimalsolutions}{[index] 1}]{solutions}{\optimalsolutions}
\pgfplotstablecreatecol[copy column from table={\knowngreedy}{[index] 1}]{greedy}{\optimalsolutions}
\pgfplotstablecreatecol[copy column from table={\knowngreedy}{[index] 3}]{greedytime}{\optimalsolutions}
\pgfplotstablecreatecol[copy column from table={\knownlocalsearch}{[index] 1}]{localsearch}{\optimalsolutions}
\pgfplotstablecreatecol[copy column from table={\knownlocalsearch}{[index] 3}]{localsearchtime}{\optimalsolutions}

% \pgfplotstablecreatecol[copy column from table={\lowtaboolist}{[index] 1}]{lowtaboo}{\optimalsolutions}
% \pgfplotstablecreatecol[copy column from table={\hightaboolist}{[index] 1}]{hightaboo}{\optimalsolutions}
% \pgfplotstablecreatecol[copy column from table={\highertaboolist}{[index] 1}]{highertaboo}{\optimalsolutions}
% \pgfplotstablecreatecol[copy column from table={\highertaboolist}{[index] 3}]{highertabootime}{\optimalsolutions}

\pgfplotstablecreatecol[copy column from table={\lowiterationlowaspiration}{[index] 1}]{lowiterationlowaspiration}{\optimalsolutions}
\pgfplotstablecreatecol[copy column from table={\lowiterationlowaspiration}{[index] 3}]{lowiterationlowaspirationtime}{\optimalsolutions}
\pgfplotstablecreatecol[copy column from table={\lowiterationhighaspiration}{[index] 1}]{lowiterationhighaspiration}{\optimalsolutions}
\pgfplotstablecreatecol[copy column from table={\lowiterationfullaspiration}{[index] 1}]{lowiterationfullaspiration}{\optimalsolutions}

% \pgfplotstablecreatecol[copy column from table={\mediumiteration}{[index] 1}]{mediumiteration}{\optimalsolutions}
\pgfplotstablecreatecol[copy column from table={\highiterationlowaspiration}{[index] 1}]{highiterationlowaspiration}{\optimalsolutions}
\pgfplotstablecreatecol[copy column from table={\highiterationlowaspiration}{[index] 3}]{highiterationlowaspirationtime}{\optimalsolutions}
\pgfplotstablecreatecol[copy column from table={\highiterationhighaspiration}{[index] 1}]{highiterationhighaspiration}{\optimalsolutions}
\pgfplotstablecreatecol[copy column from table={\highiterationfullaspiration}{[index] 1}]{highiterationfullaspiration}{\optimalsolutions}

\pgfplotstablecreatecol[copy column from table={\nolimititerationlowaspiration}{[index] 1}]{nolimititerationlowaspiration}{\optimalsolutions}
\pgfplotstablecreatecol[copy column from table={\nolimititerationlowaspiration}{[index] 3}]{nolimititerationlowaspirationtime}{\optimalsolutions}

\pgfplotstablecreatecol[copy column from table={\highiterationlowernogainlowaspiration}{[index] 1}]{highiterationlowernogainlowaspiration}{\optimalsolutions}
\pgfplotstablecreatecol[copy column from table={\highiterationlowernogainlowaspiration}{[index] 3}]{highiterationlowernogainlowaspirationtime}{\optimalsolutions}
\pgfplotstablecreatecol[copy column from table={\highiterationhighnogainlowaspiration}{[index] 1}]{highiterationhighnogainlowaspiration}{\optimalsolutions}
\pgfplotstablecreatecol[copy column from table={\highiterationhighnogainlowaspiration}{[index] 3}]{highiterationhighnogainlowaspirationtime}{\optimalsolutions}

\pgfplotstableset{
    every head row/.style={
		after row=\hline
	},
    columns/0/.style={
        column name=\textsc{Instancia},
        column type={l},
        string replace*={_}{\_},
        string type,
        assign cell content/.code={
            \pgfkeyssetvalue{/pgfplots/table/@cell content}{\texttt{##1}}
        }
    },
    columns/solutions/.style={
        column name=$MCS$,
        int detect
    },
    columns/greedy/.style={
    	column name=$greedy$
    },
    columns/localsearch/.style={
        column name=$LS$
    },
    columns/localsearchtime/.style={
        column name=$T(LS)$
    },
	% columns/lowtaboo/.style={
 %        column name=$\#E(taboo 1)$
 %    },
 %    columns/hightaboo/.style={
 %        column name=$\#E(taboo 60)$
 %    },
 %    columns/highertaboo/.style={
 %        column name=$\#E(taboo 600)$
 %    },
    columns/lowiterationlowaspiration/.style={
        column name=$i1000a0.3$
    },
    columns/lowiterationlowaspirationtime/.style={
        column name=$T(i1000)$
    },
    columns/lowiterationhighaspiration/.style={
        column name=$i1000a0.8$
    },
    columns/lowiterationfullaspiration/.style={
        column name=$i1000a1.0$
    },
    % columns/mediumiteration/.style={
    %     column name=$\#E(ite 4000)$
    % },
    columns/highiterationlowaspiration/.style={
        column name=$i2000a0.3$
    },
    columns/highiterationlowaspirationtime/.style={
        column name=$T(i2000)$
    },
    columns/highiterationhighaspiration/.style={
        column name=$i2000a0.8$
    },
    columns/highiterationfullaspiration/.style={
        column name=$i2000a1.0$
    },
    columns/nolimititerationlowaspiration/.style={
        column name=$iINFa0.3$
    },
    columns/nolimititerationlowaspirationtime/.style={
        column name=$T(iINF)$
    },
    columns/highiterationlowernogainlowaspiration/.style={
        column name=$ng 100$
    },
    columns/highiterationhighnogainlowaspiration/.style={
        column name=$ng 2000$
    },
    columns/highiterationlowernogainlowaspirationtime/.style={
        column name=$T(ng 100)$
    },
    columns/highiterationhighnogainlowaspirationtime/.style={
        column name=$T(ng 2000)$
    },
	create on use/ratiolist/.style=
		{create col/expr={(\thisrow{greedy} * 100)/ \thisrow{highertaboo} - 100}},
	columns/ratio/.style={
        column name=$\% \Delta$,
        column type={r},
        fixed, precision=2,
        postproc cell content/.append style={
            /pgfplots/table/@cell content/.add={}{\%}
        },
        fonts by sign={}{\color{red}}
    },
    % create on use/ratiolisttime/.style=
    %     {create col/expr={(\thisrow{highertabootime} * 100)/ \thisrow{greedytime} - 100}},
    % columns/ratio/.style={
    %     column name=$\% \Delta$,
    %     column type={r},
    %     fixed, precision=2,
    %     postproc cell content/.append style={
    %         /pgfplots/table/@cell content/.add={}{\%}
    %     },
    %     fonts by sign={}{\color{red}}
    % },
    create on use/ratioaspiration/.style=
        {create col/expr={(\thisrow{lowiterationhighaspiration} * 100)/ \thisrow{lowiterationlowaspiration} - 100}},
    columns/ratio/.style={
        column name=$\% \Delta$,
        column type={r},
        fixed, precision=2,
        postproc cell content/.append style={
            /pgfplots/table/@cell content/.add={}{\%}
        },
        fonts by sign={}{\color{red}}
    },
    create on use/rationogaintime/.style=
        {create col/expr={(\thisrow{highiterationhighnogainlowaspirationtime} * 100)/ \thisrow{highiterationlowernogainlowaspirationtime} - 100}},
    columns/rationogaintime/.style={
        column name=$\% \Delta$,
        column type={r},
        fixed, precision=2,
        postproc cell content/.append style={
            /pgfplots/table/@cell content/.add={}{\%}
        },
        fonts by sign={}{\color{red}}
    },
    create on use/rationogaintime2/.style=
        {create col/expr={(\thisrow{highiterationhighnogainlowaspirationtime} * 100)/ \thisrow{localsearchtime} - 100}},
    columns/rationogaintime2/.style={
        column name=$\% \Delta_2$,
        column type={r},
        fixed, precision=2,
        postproc cell content/.append style={
            /pgfplots/table/@cell content/.add={}{\%}
        },
        fonts by sign={}{\color{red}}
    }
}

% \begin{figure}[H]
% 	\centering
% 	\caption{Comparación de calidad entre distintos tamaños de lista tabú}
% 	\pgfplotstabletypeset[
% 		columns={0, solutions, greedy, highertaboo, ratiolist}
% 	]{\optimalsolutions}
% \end{figure}

% \begin{figure}[H]
%     \centering
%     \caption{Comparación de tiempos entre distintos tamaños de lista tabú}
%     \pgfplotstabletypeset[
%         columns={0, greedytime, highertaboo, ratiolista}
%     ]{\optimalsolutions}
% \end{figure}

\begin{figure}[H]
    \centering
    \caption{Comparación de calidad entre distintos umbrales de la función de aspiración con i = 1000}
    \pgfplotstabletypeset[
        columns={0, solutions, localsearch, lowiterationlowaspiration, lowiterationhighaspiration, lowiterationfullaspiration}
    ]{\optimalsolutions}
\end{figure}

Es notoria la mejora que representa la metaheurística con respecto a la búsqueda local. En la mayoría de los casos, sobretodo en los que mayor cantidad de aristas tienen como los grafos aleatorios de 150 nodos, puede apreciarse una mejora sustancial en la cantidad de aristas encontradas. Sin embargo, disminuir el valor del umbral de la función de aspiración (que significa abrir paso a soluciones con menor cantidad de aristas) no tuvo un
efecto positivo, al contrario de lo que era esperado. La cantidad de aristas encontradas no varió en lo absoluto en cuanto a la metaheurística para estos parámetros.

Es por eso que se corrió esta prueba para una cantidad de iteraciones total e iteraciones sin mejora de 2000. Para esta cantidad de iteraciones el algoritmo ya toma un tiempo de ejecución considerablemente alto, y aún así no representa una mejora clara. Únicamente para la prueba con árboles de 150 nodos puede verse una mejora, y es muy pequeña. Para el resto de las instancias la cantidad de aristas encontrada por el algoritmo bajo estos parámetros permance idéntica.

\begin{figure}[H]
    \centering
    \caption{Comparación de calidad entre distintos umbrales de la función de aspiración con i = 2000}
    \pgfplotstabletypeset[
        columns={0, solutions, localsearch, highiterationlowaspiration, highiterationhighaspiration, highiterationfullaspiration}
    ]{\optimalsolutions}
\end{figure}

Se corrió otra experimentación en donde el valor a variar es la cantidad de iteraciones. Manteniendo fijos los siguierntas parametros:

\begin{itemize}
\item Tipo de vecindad = \textit{exchange}
\item Proporción de vecinos considerada = 1\%
\item Tamaño de lista tabú = 600
\item Relacion entre valor del peso de aristas vs. tiempo = 0.5
\item Umbral de la función de aspiración = 0.3. Fue elegido este valor debido
a que se lo consideró un buen indicador, la hipótesis es que cuanto más
abierto sea el umbral hay más probabilidad de explorar soluciones que por
momento sean muy malas pero que culminen en una mucho mejor.
\end{itemize}

A continuación se exhiben los datos obtenidos.

\begin{figure}[H]
    \centering
    \caption{Comparación de calidad entre cantidades de iteraciones}
    \pgfplotstabletypeset[
        columns={0, solutions, localsearch, lowiterationlowaspiration, highiterationlowaspiration, nolimititerationlowaspiration}
    ]{\optimalsolutions}
\end{figure}

Como era de esperarse, se observa que en la mayoria de los casos se produce
una mejora a medida que la cantidad de iteraciones incrementa. Naturalmente,
al ejecutarse la búsqueda una mayor cantidad de veces existe una probabilidad
mayor de encontrar una solución con una mayor cantidad de aristas.

Sin embargo el tiempo de ejecución aumenta considerablemente a medida que
aumenta la cantdad de iteraciones, por lo que es necesario plantearse qué
es conveniente sacrificar en cada caso al momento de decidir qué combinación
de parámetros emplear. La decisión está entre tiempo de ejecución y calidad de
la solución.

\begin{figure}[H]
    \centering
    \caption{Comparación de tiempos entre cantidades de iteraciones}
    \pgfplotstabletypeset[
        columns={0, localsearchtime, lowiterationlowaspirationtime, highiterationlowaspirationtime, nolimititerationlowaspirationtime}
    ]{\optimalsolutions}
\end{figure}

La siguiente experimentación consta de variar la cantidad de iteraciones
sin mejora, permaneciendo fijos los siguientes parámetros:

\begin{itemize}
\item Tipo de vecindad = \textit{exchange}
\item Proporción de vecinos considerada = 1\%
\item Tamaño de lista tabú = 60
\item Cantidad de iteraciones total = 2000
\item Relacion entre valor del peso de aristas vs. tiempo = 0.5
\item Umbral de la función de aspiración = 0.3.
\end{itemize}

\begin{figure}[H]
    \centering
    \caption{Comparación de calidad entre cantidades de iteraciones sin mejora}
    \pgfplotstabletypeset[
        columns={0, solutions, localsearch, highiterationlowernogainlowaspiration, highiterationhighnogainlowaspiration}
    ]{\optimalsolutions}
\end{figure}

Se observa una clara ganancia entre un tiempo de parada y otro, pero también
un tiempo de ejecución elevado. En el siguiente cuadro se exhibe una
comparación entre ellos.

En este experimento $\Delta\%$ representa la diferencia porcentual entre el
tiempo de ejecución de $ng 100$ y $ng 2000$.

\begin{figure}[H]
    \centering
    \caption{Comparación de tiempos entre cantidades de iteraciones sin mejora}
    \pgfplotstabletypeset[
        columns={0, localsearchtime, highiterationlowernogainlowaspirationtime, highiterationhighnogainlowaspirationtime, rationogaintime}
    ]{\optimalsolutions}
\end{figure}

Como puede apreciarse, existe una diferencia de tiempo de ejecución entre una
corrida y otra muy grande para los casos con mayor cantidad de aristas.
Esta diferencia es mucho más grande si hacemos la comparación con la búsqueda
local.
