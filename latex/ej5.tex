\section{Ejercicio 5: Heurística de búsqueda local}

% Obviamente, el Doc está en desacuerdo y dice que la heurística golosa no es
% la mejor idea. Para justificar su postura les pide diseñar e implementar una
% heurística de búsqueda local para MCS, y desarrollar los siguientes puntos
% para, de una buena vez, convencer a Marty:
% a) Explicar detalladamente el algoritmo implementado. Plantear al menos dos
%    vecindades distintas para la búsqueda.
% b) Calcular el orden de complejidad temporal de peor caso de una iteración
%    del algoritmo de búsqueda local (para las vecindades planteadas). Si es
%    posible, dar una cota superior para la cantidad de iteraciones de la
%    heurística.
% c) Realizar una experimentación que permita observar la performance del
%    algoritmo comparando los tiempos de ejecución y la calidad de las
%    soluciones obtenidas, en función de las vecindades utilizadas y elegir,
%    si es posible, la configuración que mejores resultados provea para el
%    grupo de instancias utilizado.
\subsection{Experimentación}

	Dadas ciertas familias de grafos (arboles, ciclos, completos y bipartos) y distintos tamaños de grafos ($#(g_1)$ $n$ veces mas que $#(g_2)$, variando el $n$ entre 1 y $#(g_2)$), se querra comparar, cuanto mejor que la solución greedy es el local search, cual de las dos vecindades es mejor, que porcentaje de vencindad a mirar da un mejor resultado y a partir de cuantas iteraciones empieza a agragar pocas ó deja de agregar arristas.  

	Familia de grafos para los cuales se experimenta : 

	\begin{enumerate}
		\item Se toma $g_1$ se toma un arbol de tamaño $n_1$ ($T_{n_1}$) y como $g_2$ se toma un completo de tamaño $n_2$ ($K_{n_2}$).

	\end{enumerate}
	
	\subsubsection{Greedy vs Local Search}

	Se corrieron con viendo el $5$ porciento de la vecindad y cortando a la 5 iteracion.

    \pgfplotstableread[header=false]{../exp/ej5/known_solution_instances_search_2_exp}\knownlocalsearchdos
    \pgfplotstableread[header=false]{../exp/ej5/known_solution_instances_search_1_exp}\knownlocalsearchuno
    \pgfplotstableread[header=false]{../exp/ej4/known_sol_greedy_exp}\greedysolutions

    Primera vecindad:

    \pgfplotstableset{
        create on use/sol/.style={copy column from table={\greedysolutions}{[index] 1}}
    }

    \pgfplotstabletypeset[
        every head row/.style={
            after row=\hline
        },
        columns={0, sol, 1, 3, 4},
        columns/0/.style={
            column name=\textsc{Instancia},
            column type={l},
            string replace*={_}{\_},
            string type,
            assign cell content/.code={
                \pgfkeyssetvalue{/pgfplots/table/@cell content}{\texttt{##1}}
            }
        },
        columns/sol/.style={
            column name=$\#E(G)$,
            int detect
        },
        columns/1/.style={
            column name=$\#E(LS_1)$,
            int detect
        },
        columns/3/.style={
            column name=$T_{\mu}$,
        },
        columns/4/.style={
            column name=$T_{\sigma}$,
        }
    ]\knownlocalsearchuno


    Segunda vecindad:


    

    \pgfplotstableset{
        create on use/sol/.style={copy column from table={\greedysolutions}{[index] 1}}
    }

    \pgfplotstabletypeset[
        every head row/.style={
            after row=\hline
        },
        columns={0, sol, 1, 3, 4},
        columns/0/.style={
            column name=\textsc{Instancia},
            column type={l},
            string replace*={_}{\_},
            string type,
            assign cell content/.code={
                \pgfkeyssetvalue{/pgfplots/table/@cell content}{\texttt{##1}}
            }
        },
        columns/sol/.style={
            column name=$\#E(G)$,
            int detect
        },
        columns/1/.style={
            column name=$\#E(LS_2)$,
            int detect
        },
        columns/3/.style={
            column name=$T_{\mu}$,
        },
        columns/4/.style={
            column name=$T_{\sigma}$,
        }
    ]\knownlocalsearchdos





	\subsubsection{Porcentaje de vencindad optimo}

	Familia de grafos para los cuales se experimenta contra el greedy. Para este experimento no hay limite de iteraciones y se mira toda vecindades.

    Primera vecindad
	
    \begin{figure}[H]
        \centering
        \begin{tikzpicture}
            \begin{axis}[
                    title={},
                    xlabel={$g1$ es un ramdom tree de 400 nodos y $g2$ es un ramdom tree de 800 },
                    ylabel={Cantidad de arristas promedio},
                    scaled x ticks=false,
                    scaled y ticks=false,
                    enlargelimits=0.05,
                    width=0.5\textwidth,
                    height=0.5\textwidth,
                    legend pos=north west,
                    legend cell align=left,
                    xmin=0
                ]

                \addplot[color=black] table[x index=0,y index=1]{../exp/ej5/big_tree_vs_small_tree_neighbourhood_1_proportion};

                \legend{$T_{M}$}
            \end{axis}
        \end{tikzpicture}
        \caption{Tiempos de ejecución observados al variar los valores de $n_1$
        ($T_{co_n}$) .}
        \label{fig:exp3:var-nym-base}
    \end{figure}

    \begin{figure}[H]
        \centering
        \begin{tikzpicture}
            \begin{axis}[
                    title={},
                    xlabel={$g1$ es un completo de 400 nodos y $g2$ es un ramdom tree de 800 },
                    ylabel={Cantidad de arristas promedio},
                    scaled x ticks=false,
                    scaled y ticks=false,
                    enlargelimits=0.05,
                    width=0.5\textwidth,
                    height=0.5\textwidth,
                    legend pos=north west,
                    legend cell align=left,
                    xmin=0
                ]

                \addplot[color=black] table[x index=0,y index=1]{../exp/ej5/tree_vs_complete_neighbourhood_1_proportion};

                \legend{$T_{M}$}
            \end{axis}
        \end{tikzpicture}
        \caption{Tiempos de ejecución observados al variar los valores de $n_1$
        ($T_{co_n}$) .}
        \label{fig:exp3:var-nym-base}
    \end{figure}


    \begin{figure}[H]
        \centering
        \begin{tikzpicture}
            \begin{axis}[
                    title={},
                    xlabel={$g1$ es un ciclo de 400 nodos y $g2$ es un ciclo de 800 },
                    ylabel={Cantidad de arristas promedio},
                    scaled x ticks=false,
                    scaled y ticks=false,
                    enlargelimits=0.05,
                    width=0.5\textwidth,
                    height=0.5\textwidth,
                    legend pos=north west,
                    legend cell align=left,
                    xmin=0
                ]

                \addplot[color=black] table[x index=0,y index=1]{../exp/ej5/big_cicle_vs_small_cicle_neighbourhood_1_proportion};

                \legend{$T_{M}$}
            \end{axis}
        \end{tikzpicture}
        \caption{Tiempos de ejecución observados al variar los valores de $n_1$
        ($T_{co_n}$) .}
        \label{fig:exp3:var-nym-base}
    \end{figure}

    \begin{figure}[H]
        \centering
        \begin{tikzpicture}
            \begin{axis}[
                    title={},
                    xlabel={$g1$ es un completo de 400 nodos y $g2$ es un ciclo de 800 },
                    ylabel={Cantidad de arristas promedio},
                    scaled x ticks=false,
                    scaled y ticks=false,
                    enlargelimits=0.05,
                    width=0.5\textwidth,
                    height=0.5\textwidth,
                    legend pos=north west,
                    legend cell align=left,
                    xmin=0
                ]

                \addplot[color=black] table[x index=0,y index=1]{../exp/ej5/cicle_vs_complete_neighbourhood_1_proportion};

                \legend{$T_{M}$}
            \end{axis}
        \end{tikzpicture}
        \caption{Tiempos de ejecución observados al variar los valores de $n_1$
        ($T_{co_n}$) .}
        \label{fig:exp3:var-nym-base}
    \end{figure}


    \begin{figure}[H]
        \centering
        \begin{tikzpicture}
            \begin{axis}[
                    title={},
                    xlabel={$g1$ es un ciclo de 400 nodos y $g2$ es un random tree de 800 },
                    ylabel={Cantidad de arristas promedio},
                    scaled x ticks=false,
                    scaled y ticks=false,
                    enlargelimits=0.05,
                    width=0.5\textwidth,
                    height=0.5\textwidth,
                    legend pos=north west,
                    legend cell align=left,
                    xmin=0
                ]

                \addplot[color=black] table[x index=0,y index=1]{../exp/ej5/big_tree_vs_small_cicle_neighbourhood_1_proportion};

                \legend{$T_{M}$}
            \end{axis}
        \end{tikzpicture}
        \caption{Tiempos de ejecución observados al variar los valores de $n_1$
        ($T_{co_n}$) .}
        \label{fig:exp3:var-nym-base}
    \end{figure}

    \begin{figure}[H]
        \centering
        \begin{tikzpicture}
            \begin{axis}[
                    title={},
                    xlabel={$g1$ es un random tree de 400 nodos y $g2$ es un ciclo de 800 },
                    ylabel={Cantidad de arristas promedio},
                    scaled x ticks=false,
                    scaled y ticks=false,
                    enlargelimits=0.05,
                    width=0.5\textwidth,
                    height=0.5\textwidth,
                    legend pos=north west,
                    legend cell align=left,
                    xmin=0
                ]

                \addplot[color=black] table[x index=0,y index=1]{../exp/ej5/big_cicle_vs_small_tree_1_proportion};

                \legend{$T_{M}$}
            \end{axis}
        \end{tikzpicture}
        \caption{Tiempos de ejecución observados al variar los valores de $n_1$
        ($T_{co_n}$) .}
        \label{fig:exp3:var-nym-base}
    \end{figure}





	\subsubsection{Cantidad de iteraciones optima}