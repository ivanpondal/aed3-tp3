\section{Ejercicio 5: Heurística de búsqueda local}

% Obviamente, el Doc está en desacuerdo y dice que la heurística golosa no es
% la mejor idea. Para justificar su postura les pide diseñar e implementar una
% heurística de búsqueda local para MCS, y desarrollar los siguientes puntos
% para, de una buena vez, convencer a Marty:
% a) Explicar detalladamente el algoritmo implementado. Plantear al menos dos
%    vecindades distintas para la búsqueda.
% b) Calcular el orden de complejidad temporal de peor caso de una iteración
%    del algoritmo de búsqueda local (para las vecindades planteadas). Si es
%    posible, dar una cota superior para la cantidad de iteraciones de la
%    heurística.
% c) Realizar una experimentación que permita observar la performance del
%    algoritmo comparando los tiempos de ejecución y la calidad de las
%    soluciones obtenidas, en función de las vecindades utilizadas y elegir,
%    si es posible, la configuración que mejores resultados provea para el
%    grupo de instancias utilizado.
\subsection{Experimentación}

	Dadas ciertas familias de grafos (arboles, ciclos, completos y bipartos) y distintos tamaños de grafos ($#(g_1)$ $n$ veces mas que $#(g_2)$, variando el $n$ entre 1 y $#(g_2)$), se querra comparar, cuanto mejor que la solución greedy es el local search, cual de las dos vecindades es mejor, que porcentaje de vencindad da un mejor resultado y a partir de cuantas iteraciones empieza a agragar pocas ó deja de agregar arristas.  

	Se pasa a enumerar los experimentos.

	\begin{enumerate}
		\item Como $g_1$ se toma un arbol de tamaño $n_1$ ($T_{n_1}$) y como $g_2$ se toma un completo de tamaño $n_2$ ($K_{n_2}$).

	\end{enumerate}
	