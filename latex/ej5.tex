\section{Ejercicio 5: Heurística de búsqueda local}

% Obviamente, el Doc está en desacuerdo y dice que la heurística golosa no es
% la mejor idea. Para justificar su postura les pide diseñar e implementar una
% heurística de búsqueda local para MCS, y desarrollar los siguientes puntos
% para, de una buena vez, convencer a Marty:
% a) Explicar detalladamente el algoritmo implementado. Plantear al menos dos
%    vecindades distintas para la búsqueda.
% b) Calcular el orden de complejidad temporal de peor caso de una iteración
%    del algoritmo de búsqueda local (para las vecindades planteadas). Si es
%    posible, dar una cota superior para la cantidad de iteraciones de la
%    heurística.
% c) Realizar una experimentación que permita observar la performance del
%    algoritmo comparando los tiempos de ejecución y la calidad de las
%    soluciones obtenidas, en función de las vecindades utilizadas y elegir,
%    si es posible, la configuración que mejores resultados provea para el
%    grupo de instancias utilizado.

\subsection{Introducción}
Este ejercicio consiste en la implementación de una heurística de búsqueda
local para el problema de \acr{MCS}, que sea capaz de iterar sobre una
solución dada y mejorarla en búsqueda de óptimos locales.

Al igual que en el ejercicio anterior, asumiremos sin pérdida de generalidad
que los grafos $G_1$ y $G_2$ para los que quiere resolverse el problema son
tales que $N_1 \leq N_2$, siendo $N_1$ y $N_2$ sus cantidades de nodos
respectivas.

\subsection{Vecindades planteadas}

\subsection{Implementación}

\subsection{Complejidad temporal}

\subsection{Experimentación}
