\section{Ejercicio 8}

% Como ejercicio EXTRA y NO OBLIGATORIO pueden desafiar a Marty y el Doc,
% resolviendo el problema mediante un código en C++ que use cualquier técnica
% algorítmica o heurística que deseen pero que demore no más de 15 segundos en
% terminar. Este ejercicio debe ser resuelto en grupos de a lo sumo dos
% personas (para no poner en desventaja numérica a nuestros personajes), que
% deben, a su vez, pertenecer al mismo grupo de TP. Todos los códigos que sean
% entregados como parte de este ejercicio competirán por un premio. Para la
% competencia se armará un fixture con los grupos participantes, donde cada
% etapa se resolverá corriendo el algoritmo de cada uno para algún(os) caso(s)
% de prueba y los grupos se irán eliminando por eliminación directa, hasta
% quedar un único grupo ganador. En cada instancia, cada grupo obtendrá un
% puntaje proporcional a la cantidad de ejes del resultado que se obtenga,
% siempre y cuando sea un subgrafo de ambos grafos, o 0 (cero) en caso
% contrario o si tarda más de 15 segundos en terminar. Los inscripción a esta
% competencia cierra el 10 de junio y se realizará enviando un correo
% electrónico a algo3.dc@gmail.com con subject COMPETENCIA-TP3, informando en
% el contenido nombre y apellido de cada uno de los integrantes del grupo. Hay
% interesantísimos premios y menciones para el(los) grupo(s) ganadores.
