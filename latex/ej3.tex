\section{Ejercicio 3}

% El nuevo problema que tienen Marty y el Doc es que ahora no se ponen de
% acuerdo en qué estructura quı́mica deben utilizar para armar el condensador
% de flujos. Marty, a partir de las propiedades atómicas, ha obtenido una
% constante n y está seguro de que la solución que están buscando debe ser un
% grafo de a lo sumo n vértices (o como él lo dice, un subgrafo de algún Kn),
% mientras que el Doc ha estudiado las caracterı́sticas moleculares e insiste
% con que debe ser un subgrafo de un cografo dado. Por las dudas, y para
% evitar más conflictos entre ellos, quieren cumplir con ambas
% especificaciones, siempre maximizando la cantidad de aristas, dado que éste
% es el punto clave para su funcionamiento dentro del condensador. Por eso les
% pedimos que los ayuden diseñando e implementando un algoritmo exacto para
% MCS que tenga complejidad temporal polinomial para el caso en el que G1 es
% un cografo y G2 es un grafo completo y desarrollen los siguientes puntos que
% avalen la solución encontrada (si estamos hablando de viajar en el tiempo,
% no hay margen de error posible):
% a) Explicar detalladamente el algoritmo implementado.
% b) Calcular el orden de complejidad temporal de peor caso del algoritmo.
% c) Realizar una experimentación que permita observar los tiempos de
%    ejecución del algoritmo en función del tamaño de entrada.
